\section{Mikołaj Kołek}
\label{sec:Mikołaj}

Dodane zdjęcie: (Figure \ref{fig:cat}).

\begin{figure}[htbp]
    \centering
    \includegraphics[width=0.3\textwidth]{pictures/mikołaj.jpeg} 
    \caption{To jest kot}
    \label{fig:cat}
\end{figure}

Table~\ref{tab:mikołaj_tabela} to tabela.

\input{tables/mikołaj_tabela}

Wyrażenie matematyczne : 
\begin{align}
\lim_{{x \to \infty}} f(x) &= 2x^2 + 5x + 7 \\
g(x) &= \frac{1}{f(x)}
\end{align}

Nienumerowana lista:
\begin{itemize}
  \item Pierwszy
  \item Drugi
  \item Trzeci
\end{itemize}

Numerowana lista:
\begin{enumerate}
  \item Pierwszy numerowany
  \item Drugi numerowany
  \item Trzeci numerowany
\end{enumerate}

\setlength{\parindent}{20pt}

\section*{Krótki tekst}

\textbf{First paragraph} of a section which, as you can see, \textit{Some of the greatest \emph{discoveries} 
in science 
were made by accident.}

\textbf{Second paragraph}. As you can \underline{see} it is \textbf{\textit{indented}}.

