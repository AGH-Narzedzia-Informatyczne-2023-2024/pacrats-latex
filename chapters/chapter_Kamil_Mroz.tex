\section{Kamil Mróz}
\label{sec:Kamil}

Zdjęcie Alfiego (see Figure~\ref{fig:alfie}).

\begin{figure}[htbp]
    \centering
    \includegraphics[width=0.3\textwidth]{pictures/Kamil.jpg}
    \caption{To Alfie Solomons}
    \label{fig:alfie}
\end{figure}

Table~\ref{tab:numbers} przedstawia jakieś liczby. % Do czego służy \ref{}?

\begin{table}[htbp]
\centering
\begin{tabular}{|l|l|l|l|l|} 
 \hline
 Kolumna1 & Kolumna2 & Kolumna3 & Kolumna4 & Kolumna5 \\ \hline 
3   & 44   & 23  & 6  & 6   \\ \hline
234 & 4    & 55  & 23 & 234 \\ \hline
421 & 523  & 234 & 5  & 4   \\ \hline
23  & 5234 & 1   & 56 & 7  \\ \hline
\end{tabular}
\label{tab:numbers}
\caption{Tablea z jakimiś liczbami.}
\end{table}

Jakiś limes: \[\lim_{n\to\infty} (1+ {1\over\displaystyle n})^n = e\]

Lista nienumerowana:
\begin{itemize}
  \item pierwszy element.
  \item drugi element.
  \item trzeci element.
\end{itemize} \\

A teraz numerowana:
\begin{enumerate}
  \item trzeci element.
  \item drugi element.
  \item pierwszy element.
\end{enumerate}

\setlength{\parindent}{20pt}

\section*{Krótkie teksty}
\textbf{Pierwszy akapit} tutaj napisany jest \textit{tekst} o czymś, jeszcze nie do końca wiadomo o czym.

\textbf{Drugi akapit}. A to jest \textit{drugi} akapit, który opisuje drugi akapit i jakoś to sobie tam działa. 